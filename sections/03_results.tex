% ============================================================================
% RESULTS
% ============================================================================

\documentclass[../main.tex]{subfiles}

\begin{document}

\section{Experimental Setup}

Lorem ipsum dolor sit amet, consectetur adipiscing elit. Sed do eiusmod tempor incididunt ut labore et dolore magna aliqua. The experiments were conducted under the following conditions:

\begin{table}[H]
\centering
\caption{Experimental parameters}
\label{tab:parameters}
\begin{tabular}{ll}
\hline
\textbf{Parameter} & \textbf{Value} \\
\hline
Sample size & 1000 \\
Iterations & 500 \\
Tolerance & $10^{-6}$ \\
Method & Algorithm A \\
\hline
\end{tabular}
\end{table}

\section{Main Results}

Ut enim ad minim veniam, quis nostrud exercitation ullamco laboris nisi ut aliquip ex ea commodo consequat. Duis aute irure dolor in reprehenderit in voluptate velit esse cillum dolore eu fugiat nulla pariatur.

\begin{figure}[H]
    \centering
    % Placeholder for a figure
    \fbox{\parbox{0.7\textwidth}{\centering\vspace{3cm}
    [Your figure here]\\
    \vspace{0.5cm}
    Replace this with \texttt{\textbackslash includegraphics[width=0.7\textbackslash textwidth]\{figures/your\_figure.pdf\}}
    \vspace{3cm}}}
    \caption{Example figure showing the main results. Replace this with your actual figure.}
    \label{fig:main_result}
\end{figure}

As shown in Figure~\ref{fig:main_result}, the results demonstrate the expected behavior. Excepteur sint occaecat cupidatat non proident, sunt in culpa qui officia deserunt mollit anim id est laborum.

\section{Performance Analysis}

Sed ut perspiciatis unde omnis iste natus error sit voluptatem accusantium doloremque laudantium, totam rem aperiam, eaque ipsa quae ab illo inventore veritatis et quasi architecto beatae vitae dicta sunt explicabo.

\begin{table}[H]
\centering
\caption{Performance comparison}
\label{tab:performance}
\begin{tabular}{lccc}
\hline
\textbf{Method} & \textbf{Accuracy} & \textbf{Time (s)} & \textbf{Memory (MB)} \\
\hline
Method A & 95.3\% & 12.4 & 256 \\
Method B & 93.1\% & 8.7 & 192 \\
Method C & 97.2\% & 18.9 & 384 \\
\hline
\end{tabular}
\end{table}

Table~\ref{tab:performance} presents a comparison of different methods. Nemo enim ipsam voluptatem quia voluptas sit aspernatur aut odit aut fugit, sed quia consequuntur magni dolores eos qui ratione voluptatem sequi nesciunt.

\section{Statistical Analysis}

Neque porro quisquam est, qui dolorem ipsum quia dolor sit amet, consectetur, adipisci velit, sed quia non numquam eius modi tempora incidunt ut labore et dolore magnam aliquam quaerat voluptatem.

The mean value was found to be $\mu = 42.7 \pm 2.3$, with a standard deviation of $\sigma = 8.5$. At vero eos et accusamus et iusto odio dignissimos ducimus qui blanditiis praesentium voluptatum deleniti atque corrupti quos dolores et quas molestias excepturi sint occaecati cupiditate non provident.

\end{document}
